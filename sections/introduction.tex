An application’s architecture is continually refactored to accommodate the shifting needs of its users \cite{}. This dynamism requires
developers to collaboratively re-design the application’s compo-
nents \cite{}, using shared application-specific expertise to design
cost-effective solutions. Collaborative refactoring and design skills
are critically important in industry \cite{}, and entire development
paradigms like Agile are founded on effective collaborative design
principles.
Yet we lack a practical method to teach high-level collaborative
design skills \cite{}. While classroom-industry partnerships
provide some practice \cite{}, the months of organizing and semesters
of execution limit their repeatability and scalability. Furthermore,
these partnerships often prioritize solitary implementation, provid-
ing few opportunities to practice collaborative design.



We draw on frameworks of serious play and values at play (VAP)
to propose an activity to address this gap \cite{}. Serious play refers
to the use of game-based methods to achieve serious, educational
outcomes. Researchers have shown that serious play can improve
motivation, foster collaboration, and enhance problem-solving skills
among students \cite{}. VAP brings attention to the social values
demonstrated through play and may enable teaching interpersonal
skills in conjunction with serious play.

Professional software design practice requires both technical re-
design competencies and interpersonal skills like joining constructive discussion, active listening, and intellectual humility \cite{}. We
have designed an activity based on an analog prototype to directly
teach collaborative design and facilitate constructive discussion.
We began by running an analog prototype of the activity in
which a guest developer first presented a sketch of their applica-
tion, then directed teams of students to redesign a small component
without writing code. Teams brainstormed and refined solutions,
while the developer pointed out where solutions might fail or be
impractically expensive. Initial results were promising, with stu-
dents and instructors reporting high satisfaction. Unfortunately,
this manual intervention suffers from three severe limitations: ap-
plication overviews are difficult to create, guest developers are rare,
and problem-solving without an interactive system is difficult for
Students.

We propose a new system, the LLM-managed application overview,
to address the prototype’s limitations. From an instructor prompt,
the system generates an overview represented in class-responsibility-
collaborator cards. A design aid from Agile, these cards are optimized to comprehend and wrangle an application’s architectural details in a deck of simple plaintext flashcards without code \cite{}.
To re-capture the real-world complexity of a codebase, our system enhances each card with an emulation of common code metrics \cite{}. The instructor presents the enhanced cards and a set of design goals to the students, who individually redesign the application’s architecture by editing the cards. We approximate the
feedback from a guest developer both by heuristically estimating,
updating, and tracking effort and code metrics incurred by changes
and by highlighting changes that may break the application.
This system allows the instructor to focus on moderating discus-
sion and teaching design. After some redesigning, students return
to their teams to discuss their solutions. With instructor oversight,
teams will assemble, refine, and pitch a single solution as a Request
For Change (RFC), referencing estimated effort, altered metrics, and
rejected design choices in their write-up.
In short, the system is an enhancement of an existing design aid
that enforces the rules to a collaborative serious game. As such,
it does not grade student designs, instead deferring to instructors
to provide design goals and situational context that will be used
in evaluating teams’ write-ups (e.g. “The system needs a patch
ASAP. Minimize required effort.” or “Redesign without changing
functionality; mind the metrics.”). By the activity’s end, the students
will have practiced all the design skills necessary for a months-long
cycle of software development, without any students or instructors
manually managing any implementation details.

